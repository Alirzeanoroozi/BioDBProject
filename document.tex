% CAMM 535 Final Project — Report 1 (Single-file LaTeX Template)
% Group 4 — Schwannoma
%
% Part 0 checklist:
%  ✓ Clean, consistent report format
%  ✓ Automatic figure numbering
%  ✓ Proper figure captions (legends) + in-text referencing
%  ✓ Proper citations + references (single .tex file; no .bib)
%
% Compile (recommended):
%   pdflatex report1.tex
%   pdflatex report1.tex

\documentclass[11pt]{article}

% ---------- Page / Layout ----------
\usepackage[margin=1in]{geometry}
\usepackage{setspace}
\setstretch{1.15}
\usepackage{microtype}

% ---------- Fonts & Encoding ----------
\usepackage[T1]{fontenc}
\usepackage[utf8]{inputenc} % keep for pdflatex
\usepackage{lmodern}

% ---------- Figures / Tables ----------
\usepackage{graphicx}
\usepackage{float}          % [H] if you need strict placement
\usepackage{booktabs}
\usepackage{longtable}
\usepackage{array}

% ---------- Captions (Figure Legends) ----------
\usepackage{caption}
\captionsetup{
	font=small,
	labelfont=bf,
	labelsep=period
}
\usepackage{subcaption}

% ---------- References / Hyperlinks ----------
\usepackage[hidelinks]{hyperref}
\usepackage[nameinlink,capitalise,noabbrev]{cleveref}

% ---------- Citations (single-file; no .bib) ----------
% Simple and convenient: manual bibliography with \cite{key}
\usepackage[numbers,sort&compress]{natbib}

% ---------- Title metadata ----------
\title{CAMM 535 Final Project — Report 1\\Group 4: Schwannoma}
\author{
	Member 1: \textit{Alireza Khoeini} \\
	Member 2: \textit{Alireza Noroozi} \\
	Member 3: \textit{Ghada}
}
\date{January 2026}

% ---------- Handy commands ----------
\newcommand{\tool}[1]{\texttt{#1}}
\newcommand{\file}[1]{\texttt{#1}}
\newcommand{\db}[1]{\textbf{#1}}

\begin{document}
	\maketitle
	
	\begin{abstract}
		This report documents data acquisition and the conceptual database design for Schwannoma.
		We integrate multiple biological databases to construct a unified schema linking phenotype-related
		genes, variants, genomic coordinates, proteins, and GEO2R results.
	\end{abstract}
	
	\tableofcontents
	\newpage
	
	% =========================================================
	\section{Part 1 — Disease / Phenotype Introduction}\label{sec:intro}
	Schwannoma is a typically benign, slow-growing peripheral nerve sheath tumor that arises
	from Schwann cells, the glial cells responsible for myelination of peripheral nerves.
	Schwannomas most commonly affect cranial and spinal nerves and are frequently associated
	with the vestibular branch of the eighth cranial nerve, where they are known as vestibular
	schwannomas (also called acoustic neuromas) \cite{review_schwannoma_2020}.
	
	\subsection{Clinical characteristics}
	Clinically, schwannomas usually present as well-circumscribed, encapsulated tumors.
	Symptoms depend on tumor location and size and may include hearing loss, tinnitus,
	balance disturbances, localized pain, or neurological deficits due to nerve compression.
	Vestibular schwannomas represent the most common subtype and account for approximately
	8--10\% of all intracranial tumors \cite{vestibular_epidemiology_2019}.
	Although most schwannomas are sporadic, bilateral vestibular schwannomas are a hallmark
	feature of neurofibromatosis type 2 (NF2), a hereditary tumor predisposition syndrome
	\cite{omim_nf2}.
	\subsubsection{Annual incidence (sporadic vestibular schwannoma)}
	
	Population-based registry data suggest that sporadic vestibular schwannoma (VS) is diagnosed at an annual rate on the order of a few cases per 100{,}000 people. In a recent UK cohort registry study (2013--2016), the mean annual incidence of newly diagnosed VS was reported as \textbf{2.2 per 100{,}000 person-years}, with incidence rising strongly with age and peaking in the \textbf{60--69} year group (about \textbf{5.8 per 100{,}000 person-years}) \cite{FernandezMendez2023}. Consistent with this scale, recent US population-based surveillance (CBTRUS, 2017--2021) reports an age-adjusted incidence rate for \textbf{vestibular schwannoma of 1.52 per 100{,}000} and shows that vestibular schwannoma constitutes the majority of non-malignant nerve sheath tumors in the CNS (Figure~\ref{fig:cbtrus_vs_share}) \cite{Price2024CBTRUS}.
	
	\begin{figure}[t]
		\centering
		\includegraphics[width=0.92\linewidth]{figures/cbtrus_figure12.jpg}
		\caption{CBTRUS (US, 2017--2021): distribution of non-malignant primary brain/CNS tumors by (A) site and (B) histopathology. Panel B highlights that vestibular schwannoma represents a large fraction of nerve sheath tumors. Adapted from \cite{Price2024CBTRUS}.}
		\label{fig:cbtrus_vs_share}
	\end{figure}
	
	\subsection{Biological relevance}
	From a biological perspective, schwannomas provide an important model for studying
	tumor suppressor gene dysfunction and dysregulated cell signaling in glial-derived tumors.
	Schwann cells play a critical role in axonal support and nerve regeneration, and disruption
	of their growth control mechanisms can lead to uncontrolled proliferation.
	Schwannoma cells typically retain a benign histological appearance, yet they can cause
	significant morbidity due to their anatomical location and compressive growth pattern
	\cite{pathology_schwannoma_2018}.
	
	\subsection{Known genetic associations}
	...
	
The most well-established genetic association in schwannoma is loss-of-function mutation
or deletion of the \emph{NF2} gene, which encodes the tumor suppressor protein merlin
(schwannomin). Merlin is a key regulator of contact inhibition and multiple signaling
pathways, including Hippo, PI3K/AKT, and MAPK pathways \cite{merlin_signaling_2021}.
In sporadic schwannomas, somatic alterations of \emph{NF2} are observed in a majority of
cases, while germline mutations are characteristic of patients with NF2-associated disease.
Additional genetic and epigenetic alterations involving pathways related to cytoskeletal
organization and cell adhesion have also been reported, highlighting the molecular
heterogeneity of schwannomas \cite{genomics_schwannoma_2022}.

\begin{figure}[htbp]
	\centering
	\includegraphics[width=0.85\linewidth]{figures/nf2_merlin_pathway.png}
	\caption{\textbf{NF2/merlin signaling pathways in schwannoma development.}
		Merlin regulates multiple growth-control pathways, including Hippo, PI3K/AKT,
		and MAPK signaling. Loss of NF2 function removes growth inhibition in Schwann
		cells and promotes tumor formation. Adapted from \cite{merlin_signaling_2024}.}
	
	\label{fig:nf2_pathway}
\end{figure}

As illustrated in \Cref{fig:nf2_pathway}, disruption of merlin-mediated signaling
removes critical growth-inhibitory signals in Schwann cells.

	\newpage
	% =========================================================
	\section{Part 2 — BioMart $\rightarrow$ STRING $\rightarrow$ Variants}\label{sec:biomart-string-variants}
	% Document steps reproducibly:
	% - BioMart filters/attributes + exported file name
	% - STRING settings (first shell, max 100 interactors)
	% - Variant retrieval (SNP/indel) + mapping and issues
	
	\subsection{Gene list retrieval from BioMart}
	Tool: \db{Ensembl BioMart}.
	
	\begin{itemize}
		\item Dataset: \emph{Homo sapiens genes (GRCh38.pXX)}.
		\item Filters: phenotype/disease term (Schwannoma) or equivalent.
		\item Attributes exported: gene symbol, Ensembl Gene ID, (optional) RefSeq ID, chr, start/end.
		%\item Output file: \file{data/biomart_schwannoma_genes.tsv}.
	\end{itemize}
	
	\subsection{Network expansion using STRING}
	Tool: \db{STRING} (Human).
	
	\begin{itemize}
		\item Input: BioMart gene symbols (multiple input).
		\item Expansion: first shell, maximum 100 interactors.
		%\item Output file: \file{data/string_expanded_genes.tsv}.
	\end{itemize}
	
	\subsection{Variant retrieval and mapping}
	Tool: \db{Ensembl BioMart} (variants/dbSNP).
	Explain how you ensured each variant maps to a gene in the expanded network, and list any ambiguities.
	
	% =========================================================
	\section{Part 3 — UCSC Table Browser (RefSeq coordinates)}\label{sec:ucsc}
	Describe UCSC settings and outputs (genome build, table, attributes).
	
	% =========================================================
	\section{Part 4 — UniProt ID Mapping (Genes $\rightarrow$ Proteins)}\label{sec:uniprot}
	Describe UniProt mapping settings and downloaded attributes (UniProt IDs, length, PDB, mass, function, location, etc.).
	
	% =========================================================
	\section{Part 5 — GEO Dataset \& GEO2R}\label{sec:geo}
	Describe GEO search constraints (human, since Jan 1 2005), chosen dataset, and the GEO2R output table.
	
	% =========================================================
	\section{Part 6 — Additional Table from Another Database}\label{sec:extra}
	Add at least one additional table and explain how it links (or add a cross-reference table).
	
	% =========================================================
	\section{Part 7 — Conceptual Design (ER Diagram + Keys)}\label{sec:er}
	\subsection{Entity-Relationship diagram}
	%\begin{figure}[htbp]
	%	\centering
	%	\includegraphics[width=0.95\linewidth]{figures/er_diagram.pdf}
	%	\caption{\textbf{ER diagram of the Schwannoma database.} Cardinalities are shown on relationships.}
	%	\label{fig:er}
	%\end{figure}
	
	\subsection{Primary keys and foreign keys}
	List PK/FK per table (you can use bullets or a small table).
	
	% =========================================================
	\section{Challenges and Ambiguities}\label{sec:challenges}
	Summarize issues (ID mismatches, missing mappings, non-coding genes, isoforms, etc.).
	
	% =========================================================
	\section*{References}
	% Single-file bibliography:
	% Add items as you go; keep keys stable, cite with \cite{key}.
	\begin{thebibliography}{99}
		
	\bibitem{review_schwannoma_2020}
	Plotkin, S. R., \& Asthagiri, A. R.
	\newblock Schwannoma and vestibular schwannoma.
	\newblock \emph{The Lancet Oncology}, 21(7), e349--e359, 2020.
	
	\bibitem{vestibular_epidemiology_2019}
	Carlson, M. L., et al.
	\newblock Epidemiology of vestibular schwannoma.
	\newblock \emph{Otolaryngologic Clinics of North America}, 52(4), 607--620, 2019.
	
	\bibitem{omim_nf2}
	OMIM.
	\newblock Neurofibromatosis type 2 (NF2), OMIM Entry \#101000.
	\newblock \url{https://omim.org/entry/101000}
	
	\bibitem{pathology_schwannoma_2018}
	Rodriguez, F. J., et al.
	\newblock Pathology of peripheral nerve sheath tumors.
	\newblock \emph{Acta Neuropathologica}, 135(1), 1--19, 2018.
	
	\bibitem{merlin_signaling_2021}
	Li, W., \& Cooper, J.
	\newblock The tumor suppressor merlin and its role in NF2-associated tumors.
	\newblock \emph{Nature Reviews Cancer}, 21, 593--610, 2021.
	
	\bibitem{genomics_schwannoma_2022}
	Agnihotri, S., et al.
	\newblock Genomic and epigenomic landscape of schwannomas.
	\newblock \emph{Nature Communications}, 13, 1842, 2022.
	
	\bibitem{merlin_signaling_2024}
	Duo Xu, Shiyuan Yin \& Yongqian Shu
	\newblock NF2: An underestimated player in cancer metabolic reprogramming and tumor immunity.
	\newblock \emph{npj Precision Oncology}, 2024.
	
	\bibitem{FernandezMendez2023}
	R.~Fernández-Méndez, Y.~Wan, P.~Axon, and A.~Joannides,
	``Incidence and presentation of vestibular schwannoma: a 3-year cohort registry study,''
	\emph{Acta Neurochirurgica}, vol.~165, no.~10, pp.~2903--2911, 2023.
	doi:10.1007/s00701-023-05665-9.
	
	\bibitem{Price2024CBTRUS}
	M.~Price \emph{et al.},
	``CBTRUS Statistical Report: Primary Brain and Other Central Nervous System Tumors Diagnosed in the United States in 2017--2021,''
	\emph{Neuro-Oncology}, vol.~26, Suppl.~6, pp.~vi1--vi85, 2024.
	doi:10.1093/neuonc/noae145.
		
		\bibitem{string_db}
		STRING Consortium.
		\newblock STRING database: known and predicted protein-protein interactions.
		\newblock Accessed: Jan 2026.
		
		\bibitem{ensembl_biomart}
		Ensembl.
		\newblock Ensembl BioMart.
		\newblock Accessed: Jan 2026.
		
		\bibitem{ucsc_table_browser}
		UCSC Genome Browser.
		\newblock UCSC Table Browser tool documentation.
		\newblock Accessed: Jan 2026.
		
		\bibitem{uniprot}
		UniProt Consortium.
		\newblock UniProt: the universal protein knowledgebase.
		\newblock Accessed: Jan 2026.
		
		\bibitem{geo}
		NCBI.
		\newblock Gene Expression Omnibus (GEO) and GEO2R.
		\newblock Accessed: Jan 2026.
		
	\end{thebibliography}
	
\end{document}
